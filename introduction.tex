% Définir un logiciel vidéo non linéaire, non intrusif,

% Définir le terme de logiciel libre

% Les questions que l'on doit susciter:

%   * Pourquoi voudrait-on utiliser des logiciels libres?

%   * Quels avantages apporteraient-t-ils?

%   * %Le logiciel libre est-il techniquement capable de

%   répondre aux attentes de professionnels?

%   * Comment des entreprise pourraient rentabilser le développement

%   des logiciels libres?

\setcounter{page}{1} \newpage \chapter*{Introduction}

\paragraph{}

Depuis le début des années 1990, le montage de
productions audiovisuelles professionnelles est réalisé dans la grande
majorité des cas de manière informatique. Les logiciels utilisés
sont variés, chaque phase de la post-production est théoriquement
effectuée par un logiciel spécialisé. La partie montage à proprement
parlé, c'est à dire, la phase durant laquelle on met bout à bout
les différentes images, est réalisée avec un logiciel dit de montage
vidéo non linéaire et non destructif. Cela signifie que les logiciels
permettent l'accès de manière instantanée à n'importe quel instant aux
fichiers sources (non linéaire), et le montage est effectué sans aucune
incidence et aucune modification de ces mêmes fichiers (non destructif).

\paragraph{}

A l'heure actuelle, le marché des logiciels de montages vidéo
professionnels est dominé par quelques entreprises commerciales qui ont
imposé leurs technologies sur le marché sans s'occuper (ou presque)
de la %compatibilité entre ces logiciels.  Cela oblige donc
les utilisateurs %une fois leur choix effectué , de poursuivre
dans ce même choix pour tous les logiciels %
%pour le processus de post-production. Les utilisateurs de ces suites
de logiciels sont donc très dépendants de leurs fournisseurs et nous
pensons qu'une autre manière d'envisager la création de logiciels de
montages pourrait permettre au marché de devenir plus concurrentiel et
plus centré sur les besoins réels des utilisateurs.

\paragraph{} %Les``Logiciels Libres'' se caractérisent par le fait que
que le code source n'est pas accessible et utilisable exclusivement par
une entreprise éditrice, mais%qu'il est accessible à tous dès qu'
il a été demandé par une personne ayant obtenu une version
binaire de ce code. %Ceci apporte plus de souplesse à
l'utilisateur %avec pour principales raisons:

\begin{itemize}

  \item {Execution du code binaire sans aucune condition limitante}

  \item {Étude du code source}

  \item {Possibilité d'adapter le code source à ses besoins, sous
  certaines
    conditions définis par la licence}

\end{itemize}

\paragraph {}

Les logiciels libres ou 'open source' sont utilisés dans de nombreux
secteurs de l'informatique quel que soit le type d'entreprise. Ils
permettent de répondre à de nombreuses problématiques de l'informatique
moderne, que ce soit au niveau des servers (où leur part de marché
représente entre 60 et 70%) ou au niveau des clients (que
ce soit poste de travail ou Smartphone).

\paragraph {}

En terme économique, l'univers du logiciel libre a su s'intégrer
au marché, les entreprises vendant principalement du service et du
consulting. Ce marché est en croissance très importante depuis plusieurs
années avec un taux de croissance de l'ordre 66\% pour l'année 2008
selon zdnet \footnote{La France est devenue « un pays phare pour le
logiciel libre »: http://tinyurl.com/france-logiciel-libre}.

\paragraph{}

Il est dans ce contexte intéressant de voir quelle part de marché les
logiciels libres ont ou pourront avoir dans le secteur professionnel de
la production audiovisuelle .

Pour cela, nous allons essayer de connaître les besoins de ces monteurs
professionnels en étudiant les fonctionnalités dont ils disposent
dans les logiciels existants, mais aussi en les interviewant et en
leur demandant de les évaluer.  Il conviendra aussi de savoir si ces
logiciels libres présentent un intérêt pour le marché de l'édition
professionnelle . Nous allons chercher %comprendre si les logiciels
libres répondent, où pourront répondre dans le futur, aux enjeux
que présentent les logiciels professionnels de montages vidéo non
destructifs non linéaires en terme technologique. Dans ce but, il sera
important d'analyser les différents logiciels existants mais aussi, les
différentes technologies libres permettant de faire du montage vidéo .
