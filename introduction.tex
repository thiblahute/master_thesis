% Définir ce qu'est un logiciel vidéo non linéaire, non intrusif,

% Définir le term de logiciel libre

% Les questions que l'on doit susciter:

%   * Pourquoi voudrait on utiliser des logiciels libres

%   * Quels avantages Apporteraient t il?

%   * Est ce que le logiciel libre est techniquement capable de

%   répondre aux attentes de pro?

%   * Comment des entreprise pourrai gagner de l'argent en développant

%   des logiciels libres?

\setcounter{page}{1}
\newpage \chapter{Introduction}

\paragraph{}

Depuis la fin des années 1980 debut des années 1990, le montage
de production audiovisuel professionnel est réaliser dans la grande
majorité des cas de manière informatique. Les logiciels utilisé à ce
but sont varié, chaque phase de la post-production est théoriquement
effectué par un logiciel adapté. La partie de montage à proprement
parlé, c'est à dire, la phase durant laquelle on met bout à bout
les différentes images est réalisé avec un logiciel dit de montage
vidéo non linéaire et non destructif. Celà signifie que les logiciels
permettent l'accès de manière instantané à n'importe quel instant des
fichiers sources (non linéaire), et le montage est effectué sans aucun
incidence et aucun modification de ces mêmes fichier (non destructif).

\paragraph{}

A l'heure actuel, le marché des logiciels de montages vidéo
professionnel est dominé par quelques entreprises commercials ayant
imposé leurs technologies sur le marché sans permettre ou presque la
communications entre les logiciels des différent acteurs.  Celà induit
qu'une fois une solution logiciel choisi, il est nécessaire d'utiliser
tout les logiciel de la même suite logiciel dans le processus de
post-production. Celà et le fait que les utilisateurs de ces suites
logiciels sont dépendant des éditeurs des logiciels nous laisse à
penser qu'une autre manière d'envisager la création de logiciel de
montages pourrai permettre au marcher de devenir plus concurrentiel et
plus axés sur les besoins réel des utilisateurs.

\paragraph{}

Cette autre manière de faire est ce que l'on appel ``Logiciel Libres''.
C'est à dire qu'au lieu que le code source soit accessible et utilisable
par une l'entreprise éditrice, celui-ci est accessible par tous à
partir du moment où il a été demandé par une personne ayant obtenu
une version binaire de ce code. Cette manière de faire donne plus de
liberté à l'utilisateur parmi lesquelles:

\begin{itemize}

  \item {Execution du codé binaire sans aucune conditions limitantes}

  \item {Étude du code source}

  \item {Possibilité d'adapter le code source à ses besoins, sous
  certaines
    conditions définis par la license}

\end{itemize}

\paragraph {}

Les logiciels libre où 'open source' sont utilisés dans de nombreux
secteurs de l'informatique par tout types d'entreprise. Il permettent
de répondre à de nombreuse problématiques de l'informatique moderne,
que ce soit au niveau des servers (ou leur part de marché représente
entre 60 et 70\% du marché) où au niveau des clients (que se soit
poste de travail où Smartphone).

\paragraph {}

En terme économique, l'univers du logiciel libre a su s'intégrer
au marché, les entreprise vendant principalement du service et du
consulting. Ce marché est en très grande croissance depuis plusieurs
années avec par exemple, une croissance de 66\% en 2008 d'après zdnet
\footnote{La France est devenue « un pays phare pour le logiciel libre
»: http://tinyurl.com/france-logiciel-libre}.

\paragraph{}

Il est dans ce context intéressant de voir quel part de marché les
logiciels libre ont ou pourrons avoir dans le secteur de la production
audiovisuel professionnel.

Pour cela, nous allons poser différentes questions parmi lesquels
celle de savoir quelles besoins les monteurs professionnel
on effectivement. Dans ce but, nous allons voir quels sont les
fonctionnalités dont les professionnels disposent dans le différents
logiciels, mais aussi interviewer des professionnels du montage afin
de connaitre leurs points de vu sur ces différents logiciel. Aussi,
il conviendra de savoir si le marché de l'édition professionnel a
un intérêt à utiliser des logiciels libres. Nous allons chercher à
savoir si les logiciels libre répondent, où pourront répondre dans
le future aux enjeux que posent les logiciels de montages vidéo non
destructifs non linéaires professionnel en terme technologique. Dans
ce but, il sera important d'analyser les différent logiciels existant
mais aussi, les différentes technologie libre permettant de faire du
montage vidéo existantes.
