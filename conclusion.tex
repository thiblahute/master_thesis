\newpage \chapter{Conclusion}

\paragraph{}

Tout au long de ce document, nous avons identifié les besoins réels
des professionnel du montage vidéo, nous nous sommes efforcer de rendre
compte de la diversité du marché aussi bien en terme de format de
vidéo que de structures au sein de laquelle ce travail de montage est
effectué. Nous avons  cherché à connaitre la vision des professionnels
concernant les outils qu'ils utilisent (à travers d'interview), afin
de determiner si les logiciels et technologies open source peuvent ou
pourront dans le future avoir une place dans le monde de la production
audiovisuel de qualité professionnel.

\paragraph{}

Au final, ce document permet de montrer que le marché du logiciel
d'édition vidéo est un marché ou peu d'entreprise ont réussi à
jouer un rôle. Les professionnels du montage vidéo ne sont que très
peu friand de changement dans leurs habitude quand il est question de
logiciel de montage.

\paragraph{}

Nous avons pu constater que les logiciels utilisés par les professionnels
sont dans la plupart des cas (voir dans tout les cas) beaucoup plus
puissant, configurable que ce dont ils ont besoins. Celà est est
un défaut relevé par tous les professionnel et ceux, pour tout les
logiciel du marché. Ceux-ci considèrent qu'ils sont ralentit en terme de
productivité par le fait que la complexité d'utilisation des logiciel,
qui n'est pas nécessaire dans les faits.

Notre analyse du marché nous a montré qu'actuellement un seul logiciel
libre est présent sur le marché professionnel (Cinelerra), et celui-ci
se positionne sur un créneaux très spécifique qui est celui des grosse
entreprises, et des institutions publique.

\paragraph{}

Nous nous sommes rendu compte qu'affin d'analyser le marché des
logiciel d'édition vidéo, il convient  de faire une distinguer les
entreprises par leur taille. C'est à dire que la notion de TPE, PME
et grand entreprises peuvent être utilisés. En effet ces notion ont
des conséquences direct sur ce que les entreprise vont recherché dans
pour leur choix de logiciel de montage.  Nous avons constaté que pour de
nombreuse  grandes entreprises, la dépendance vis-à-vis des éditeurs
de logiciel est quelque chose qu'il convient d'éviter au maximum, et
ce paramètre a un point important dans leur choix de logiciels, et en
particulier de logiciel de montage vidéo. Le choix de logiciels libres
permettrai de garantir cette indépendance, mais il convient de savoir
si ceux-ci sont à la hauteur. En ce qui concerne les entreprises de
tailles moyennes, celles-ci ont pour grand objectifs, de s'assurer leur
avenir. Pour ce, elles évite les prises de risque, et dans le cadre du
choix de logiciel de montage vidéo, cela signifie souvent utiliser les
logiciels leaders sur le marché. Celà leur permet de garantir qu'ils
peuvent trouver de la mains d'oeuvre compétente, et avoir des logiciels
considéré comme ``de qualité''. Dans ce cadre, les logiciels libres ne
font à l'heure actuel pas partie des solutions envisageable. En ce qui
concerne les petites entreprises, les points essentiels sont: le prix du
logiciels ainsi que leur demarcation sur le marché. Sur ces deux points,
les logiciels libre peuvent apporter de réels avantages puisqu'ils sont
tous gratuitement téléchargeable, et librement modifiable.


\paragraph{}

Nous avons donc analysé le marché des logiciels libres de deux points
de vues.

Premièrement d'un point de vue technique, afin de comprendre ce que les
différentes technologies permettent de faire, de juger de leur qualité,
et voir quelle potentiel elles ont. Nous nous somme rendu compte qu'il
n'y a pas de réel compétition sur ce segment. Deux technologies avec des
visions du problème différentes existent. MLT semble vouloir répondre
au problèmes posé par l'édition vidéo et broadcasting de la manière
la plus simple possible. Alors que GStreamer a un objectif totalement
différent qui est de répondre aux plus de problèmes posés par le monde
du multimedia possible.  Nous nous somme rendu compte d'après cette
analyse que le logiciel cinelerra est techniquement très différent
des autre projets, c'est à dire, qu'il s'agit d'un énorme projet,
avec un code monolithique\index{monolithique} absolument non documenté
qui est maintenu par une seul entreprise et non par une communauté.

\paragraph{}

Deuxièmement d'un point de vu des communautés alors que nous avons
pu nous rendre compte de l'état de santé des différents projets pour
analyser leur potentiel, et leur evolution possible et en cours.

C'est alors qu'il est apparu évident que le projet Cinelerra est
uniquement le projet d'une entreprise (Heroine Virtual) qui vise
à répondre aux besoins d'un segment du marché de l'édition
professionnel. L'analyse du marché nous a montrer que Cinelerra
est inférieur sur de nombreux points en comparaison des concurrents
commerciaux.  Mais malgré ce fait, ce projet a su conquérir une partie
du marché visée. Cela montre de manière évidente qu'au moins une
partie du marché de l'édition vidéo professionnelle est intéresser
par les avantages qu'offrent les logiciels libres.

Nous avons aussi vu que les communauté GStreamer et MLT sont active,
mais du fait de la conception technique de cet dernier, ca communauté
est beaucoup plus petite. Ce framework permet de répondre à la plupart
des besoins des professionnels du montage et broadcasting mais semble
limitant de par ca conception dans de nombreux context.  Nous avons
nous nous sommes rendu compte de la place prépondérante de GStreamer
sur le marché des framework multimedia. Ce logiciel est soutenu par de
nombreuses entreprise de renom. Mais aussi, nous avons constaté que
le coté montage vidéo n'est pas la force de ce framework à l'heure
actuel, bien que des efforts soient fait dans ce domaine, en particulier
avec création de la librairie gst-editing-services.

\paragraph{}

Pour conclure, les technologies de montage vidéo libres sont a l'heure
actuel dans une certaines mesure capable de répondre aux besoins des
professionnels, mais de nombreuse limitations existent toujours pour
que cela n'arrive réellement. A par sur le marché niche dans lequel
Cinelerra a sue se positionner, les logiciels libre à proprement parler
ne sont actuellement pas capables de satisfaire les professionnels,
soient par leur manque d'ergonomie et documentation, soit par leur manque
de fonctionnalités.

\paragraph{}

Nous avons aussi constater qu'affin de concourir face aux grand acteur
du marché, il n'est pas nécessaire d'avoir une parité en terme
de fonctionnalité avec ces logiciels, les différents interviewer
considérant utiliser entre 20 et 40 pour cent des fonctionnalités
seulement. Dans ce but, il apparait plus important se concentrer
sur les fonctionnalité essentiels qui on été mis en avant au sein
de ce document et faciliter le processus le processus de montage en
simplifiant l'usage du logiciel. Ce fait nous permet de croire que les
logiciels libre sont potentiellement capable de répondre à des besoins
de professionnels.  Mais le manque d'entreprise faisant la promotion,
le support et la formation pour ces logiciels est un facteur limitant
concernant l'adoption des technologies libres en milieu professionnel
plus difficile.

\paragraph{}

Il devient alors intéressant de savoir si le développement actuel des
logiciels libres dans les différents domaines de l'informatique, et
le fait que de nombreuses entreprises soutiennent leur développement
de manière très importante va permettre qu'un de ces logiciels ce
développe de manière a ce qu'il soit adopté par le milieu de l'édition
vidéo? La liberation de lightworks risque de jouer un rôle sur ce
marché, allons nous voir apparaitre un standard libre de l'édition
vidéo à travers de ce logiciel?
