\newpage \chapter*{Conclusion}

\paragraph{}

Tout au long de ce document, nous avons identifié les besoins réels
des professionnels du montage vidéo. Nous nous sommes efforcés de
%décrire la diversité du marché aussi bien en terme de format
de vidéo que de structures au sein de laquelle ce travail de montage
est effectué. Nous avons  cherché à connaître l'%avis des
professionnels sur les outils qu'ils utilisent (grâce à des interviews),
ceci afin d'évaluer si les logiciels et les technologies "open source"
peuvent ou pourront dans le futur avoir une place dans le monde de la
production audiovisuelle %professionnelle.

\paragraph{}

Au final, ce document permet de montrer que le marché du logiciel
d'édition vidéo est un marché où seules quelques entreprises
ont réussi à jouer un rôle. Nous avons aussi vérifié que les
professionnels du montage vidéo n'aiment pas changer de logiciel de
montage, et restent avec leurs habitudes tant que la maintenance est
assurée.

\paragraph{}

Nous avons pu constater que les logiciels utilisés par les professionnels
sont dans la plupart des cas (voire dans tout les cas) beaucoup plus
puissants, configurables que ce dont ils ont besoin. C'est un défaut
relevé par tous les professionnels pour tous les logiciels existants
sur le marché. Ils considèrent que leur productivité n'est pas à
son optimum essentiellement à cause de la complexité d'utilisation,
cette complexité provenant du surdimensionement en terme de nombre de
fonctionnalités de ces logiciels.

Notre analyse %nous a montré qu'actuellement un seul
logiciel libre est présent sur le marché professionnel (Cinelerra),
et celui-ci se positionne sur un segment très spécifique qui est celui
des entreprises% de taille importante, et des institutions publiques.

\paragraph{}

Nous avons réalisé que pour analyser le marché des logiciels
d'édition vidéo, il convient de distinguer les entreprises par leur
taille en utilisant les notions de TPE, PME et grandes entreprises. En
effet ces notions ont des conséquences directes sur les critères
de choix de logiciels.  Nous avons constaté que pour de
nombreuses  grandes entreprises, la dépendance vis-à-vis des éditeurs
de logiciels est quelque chose qu'il convient de minimiser, et ce
paramètre représente un point important dans leur choix,
et en particulier pour les logiciels de montage vidéo. 
Le choix de logiciels libres permettrait de garantir cette indépendance, mais il
convient de savoir si à toutes leurs exigeances . 
%
%
%En ce qui concerne les entreprises de tailles moyennes, celles-ci ont pour objectif principal d'
%sssurerleur pérennité. Elles évitent donc de prendre des risques,
et dans le cadre du choix de logiciels de montage vidéo, elles iront
souvent vers des produits leaders sur le marché. Elles trouveront ainsi
plus facilement du personnel compétent, et auront à leur disposition
des logiciels considérés comme ``de qualité''. Dans ce cadre, les
logiciels libres ne %font pas partie à l'heure actuelle des solutions
envisageables. 
%
%
%Les petites entreprises quant à elles recherchent des logiciels originaux et peu coûteux.
Sur ces deux points, les logiciels libres peuvent apporter de
réels avantages puisqu'ils sont tous gratuitement téléchargeables,
et librement modifiables.


\paragraph{}

Nous avons donc analysé le marché des logiciels libres %sous deux points
de vues.

Premièrement d'un point de vue technique % afin de comprendre ce que les
différentes technologies permettent de faire.%
% Nous nous sommes rendus
compte qu'il n'y a pas de réelle compétition sur ce segment. Deux
technologies avec des visions du problème différentes existent. MLT
semble vouloir répondre aux problèmes posés par l'édition vidéo
et broadcasting de la manière la plus simple possible. En revanche
l'objectif de GStreamer est totalement différent car il essaie de
répondre au plus grand nombre de problèmes posés par le monde
du multimedia.
%Notre analyse a également mis en évidence que le logiciel
Cinelerra est techniquement très différent des autres projets: il
s'agit d'un projet %important avec un code monolithique\index{monolithique}
absolument non documenté qui est maintenu par une seule entreprise et
non par une communauté.

\paragraph{}

Deuxièmement d'un point de vue des communautés afin de rendre compte
de l'état de santé des différents projets mais aussi %d'analyser leurs
potentiels et leurs évolutions possibles %.
%L'étude a mis en évidence que le projet Cinelerra est
uniquement le projet d'une entreprise (Heroine Virtual) qui vise
à répondre aux besoins d'un segment du marché de l'édition
professionnelle. L'analyse du marché nous a montré que Cinelerra
est inférieur sur de nombreux points à ses concurrents commerciaux.
Mais malgré cela, ce projet a su conquérir une partie du marché
visé. Cela montre que les avantages qu'offrent les logiciels libres
intéressent une partie du marché de l'édition video professionnelle.


Nous avons aussi vu que les communautés GStreamer et MLT sont actives,
mais compte tenu de la conception technique de ce dernier, sa communauté
est beaucoup plus petite. Ce framework permet de répondre à la plupart
des besoins des professionnels du montage et de broadcasting mais semble
limitant par sa conception dans de nombreuses situations.  Nous avons
apprécié la place prépondérante de GStreamer sur le marché des
framework multimedia. Ce logiciel est soutenu par de nombreuses entreprises
de renom. Mais nous avons %également constaté que la partie montage vidéo
n'est pas %actuellement le point fort de ce framework, bien que des
efforts soient faits dans ce domaine, en particulier avec la création de
la librairie gst-editing-services.

\paragraph{}

Pour conclure, les technologies de montage vidéo libres sont à l'heure
actuelle %capables dans une certaines mesure de répondre aux besoins des
professionnels, mais de nombreuses limites existent toujours. %Mis à part
sur le marché niche sur lequel Cinelerra a su se positionner, les
logiciels libres à proprement parlé ne sont actuellement pas capables
de satisfaire les professionnels, soit par leur manque d'ergonomie et
documentation, soit par leur manque de fonctionnalités.

\paragraph{}

Nous avons aussi constaté que pour concurrencer les grands acteurs du
marché, il n'est pas nécessaire de proposer une parité en terme de
nombre de fonctionnalités , les différents interviewés ont déclaré
n'utiliser que 20 à 40 pour cent des fonctionnalités proposées. Il
apparait ainsi plus important se concentrer sur les fonctionnalités
essentielles qui ont été %décrites dans ce document et
faciliter le processus de montage en simplifiant l'usage du logiciel. Ceci
nous permet de %penser que les logiciels libres sont potentiellement
capables de répondre à des besoins de professionnels.  Mais le manque
d'entreprises faisant la promotion, le support et la formation pour ces
logiciels est un facteur limitant.

\paragraph{}

On peut se demander si le développement actuel des logiciels libres
dans les différents domaines de l'informatique, et le soutien très
important de ce développement par de nombreuses entreprises ne vont %pas
permettre à un de ces logiciels de se développer et de s'imposer dans
le milieu de l'édition vidéo.  La libération de lightworks peut jouer
un rôle sur ce marché: allons nous voir apparaitre un standard libre
de l'édition vidéo à travers ce %type logiciel?
