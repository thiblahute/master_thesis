\newpage\section*{Abstract: Are Open source video editing softwares
a possible component of the post-production video industry ecosystem?}

\paragraph{}

Nowadays we are seeing Free and Open Source software projects become
increasingly popular and prevalent worldwide. This is true of many
domains where computer science is influential. But there are still many
fields where this is not the case, such as the video editing industry,
where the market is controlled by a few big commercial companies.

\paragraph{}

This document focuses on analyzing the professional video edition
market. The analysis is done in two parts: first, the current market of
video edition, and then the open source technologies aiming at fulfilling
professional video edition needs.


\paragraph{Focus on professionals' needs}

\paragraph{}

In this first part of the document, the goal is to ascertain the features
professionals actually use. This part aims at drawing a picture of the
current state of the video edition market, not focusing on any type of
production, but rather trying to understand what needs professional have
depending on the kind of production (i.e: Film, show, documentary\ldots)
they are editing.

To do so, we interviewed professionals, making sure to get input from
people coming from as many fields of video edition as possible. This,
in order to get a clear picture of what professionals expect from a video
editing tool.  This is particularly useful to determine the key features
per type of video content.  We also defined to what extent stability of
the tool is important.

\paragraph{}

To be able to determine what features are essentials, we divided the
market in two. The first part is the world of the post production market,
and the other one is the broadcasting market. We noticed that as features
are concerned, those two markets are so different that they needed to
be analyzed separately.

\paragraph{Focus on existing video editors}

\paragraph{}

Afterwards, we focused on the current market, analyzing the few software
projects that are currently being used by professionals. As this
document is about free and open source software, those were included
in the analysis even though, apart from Cinelerra that is being used
by professionals in a niche market, they don't actually deserve to be
part of it. During this part we noticed that in term of applications,
open source software is generally far from being at the same level of
featureset and usability as their proprietary counterparts.

\paragraph{}

We then focused on analyzing the reasons why the post-production industry
would actually be interested in using Libre software. Those main reasons
would be:

\begin{itemize}

  \item {More independence from the software vendor}

  \item {Possibility to get involved in the software development
  process}

   \item {Cost reduction}

   \item {Possibility to adapt the software to the company particular
   needs}

\end{itemize}

\paragraph{Analysis of the libre software market}

\paragraph{}

We then focused on analyzing the reasons why the post-production industry
would actually choose Free and open source softwares. After determining the
needs of professionals, we focuson the open source video editing market in
terms of "to what point the community has brought the different video editors
projects". Thispart underlined the fact that a lot of open source projects are
aiming at filling the gap between the commercial world and the free and open
source one. The problem being that instead of succeeding as one, those
projects have not managed to get a consensus, and thus the efforts could
not be combined in a fully collaborative manner.

\paragraph{Overview of the Foss technologies}

\paragraph{}

In order to find out the feasibility of free and open source video editing
software succeeding into the professional market, we needed to analyze
thoroughly the underlying technologies that support those projects.
In this part we decided to focus on the three main open source video
editor projects: Cinelerra, Kdenlive and PiTiVi.  This is where the two
main frameworks, GStreamer (which is used by PiTiVi) and MLT (used by
Kdenlive), have been analyzed. We also analyzed the Cinelerra codebase
in order to be able to compare it to others, although its architecture
is so different that we found the comparison to be meaningless. In terms
of frameworks, we noticed that whereas MLT focuses on the video edition
problem, GStreamer is much more generic and video edition is but a small
part of its target uses, and not the best supported one.

\paragraph{Analysis of the communities}

\paragraph{}

After focusing on technological part of the projects, we worked on
analyzing the communities that are actually driving those projects. This
had to be done in order to evaluate how healthy those projects are and
how they are currently being developed. During this analysis, we noticed
that those three technologies are being developed quite actively by
communities. Those communities are pretty different in terms of size and
the way they are being driven. In all cases a few companies are involved
in the development of the underlying technologies. But the big problem
is that no company is actively backing the editing software that uses
those underlying technologies. For example, there is no company actively
trying to develop and market the Kdenlive project,

\paragraph{Potential and plans}

\paragraph{}

After noticing what the needs of professionals are and analyzing the free
an open source market, we focus on understanding the current shortcomings
of open source software and technologies.  We then evaluated their
potential and tried to figure out what ways the communities should take
in order to get marketshare in the post-production professional industry.

\paragraph{}

To summarize, the current professional video editing market is still
mostly an uncharted territory as far as open source technology is
concerned. The video editing industry represents a tremendous business
opportunity to foster the development of an intuitive and powerful
video editing suite, if only concerted efforts were made to achieve that
goal quickly.
