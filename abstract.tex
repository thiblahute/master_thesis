\chapter*{Abstract: Open source video editing software, part of the
        post-production industry ecosystem?}

\paragraph{}

Nowadays we are seeing more and more free and open source projects getting
pretty popular and used worldwide. This is true in many domains where
computer science is influential. But there are still lots of computer
science domains where Free and Open Source Software are yet to be adopted
by the industry, which seems to be the case of the video edition software,
where the market is controlled by a few big commercial companies.

\paragraph{}

This document concentrate on analyzing the professionals
video edition market. This analysis is done focusing on two parts:
first, the current market of video edition, and then the open source
technologies aiming at fulfilling professionals video edition needs.

\paragraph{}

In this first part of the document, the goal is to detect the features
professionals actually use, which means that we had to interview professional
of the sector. This part aims at drawing a picture of the current state of
the video edition market, trying to not to focus on any kind of production
but, at contrary, trying to understand what needs professional have depending on
the kind of production (i.e: Film, show, documentary\ldots) they are editing. 

\subsection*{Focus on professional video editors} \paragraph{}
    After describing the video edition world, and how it is currently
    evolving, I will focus on the main video editing usage this documents
    aims at analyzing: ``Video editors for Professional video content
    creators''. Se concentrer sur dreamworks \cite{RobinRowe2001}

\subparagraph{}
    To do so, the first thing is to define what professional means. I
    will also concentrate on what these people expect from a video
    editing tool.  What are the main feature they need, and this
    should also be done making sure to distinguish the different kind
    of professional video editor that exist. I will also define to what
    extent stability of the tool is important. In this part I will get
    started by introducing Interviews from people working in this domain,
    being sure to get an as large as possible point of view.

\subsection*{State of the art} \paragraph{}
    This is where I will concentrate on drawing the State of the art in
    the open source world. This is a  technical report of the existing
    solutions.  The idea is to investigate existing technologies and
    check what use cases those technologies can fulfill, and what they
    can not. I will also see what are the plans of the developers of
    those technologies, make an analysis of the communities, and see
    how much support they have from companies. In this part, I will of
    course also analyze the video editors using those technologies and
    their communities. The analysis of communities, the way they work,
    their efficiency should be an important point of this part.

\subsection*{Focus on commercial solutions} \paragraph{}
    Right after analyzing the state of open source solutions and
    technologies, I will concentrate on what is going on in the commercial
    world. I will make an analysis of the main solutions, see how they
    fill the different needs this document describes. I will analyze
    the market and see if there is a real need for other solutions to
    come up, and what we should expect from them.

\subsection*{Comparison of the two worlds} \paragraph {}
    Now that I have a description of the two world, I will have to
    compare them, focusing on the topic I want to tackle: professional
    video editors. This comparison should take into consideration what
    I got from the interviews. This should be technical when possible,
    but very user oriented, which means that the user-friendliness of the
    solutions should have been described, and will be taken into account.

%\subsection*{Open source lacks} %\paragraph{}
    %I will then analyze the lacks open source solutions suffer from,
    %and discuss the ways to get it solved. In this part, I will analyze
    %the solutions that could be found to get more people involved in
    %communities, users, as well as developers, and how companies could
    get %money from it.

%\paragraph{At the end of this document,} you should have a good
%overview of the video editing world, understand what FOSS can offer
to %professionals in this domain. You should also understand what is
the %current state of libre softwares is now, where they are going,
and how %the communities are driving those projects.
