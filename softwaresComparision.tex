\newpage
\section{Comparaison des principaux logiciels présent sur le marcher de
l'édition vidé professionnel, et analyse des manques et risques du marché}

\paragraph{}
  Il conviendra d'analyse en profondeur les logiciels existants, qu'il soit
  propriétaires ou libre. Cette partie à pour but de rendre compte de
  l'état actuel du marcher des logiciels d'édition vidéo qui on pour principal
  public les professionnelles de l'édition vidéo. Cette étude étant principalement portée
  sur le logiciel libre, ceux-ci seront évidemment inclus dans cette analyse
  bien que l'on puisse considérer que de par leur manque de maturité,
  n'y ont pas totalement leur place. Dans cette optique, on se doit d'analyser
  points clés des logiciels, tout d'abord on comparera les fonctionnalités des
  logiciels, la manière dont elles sont géré, et essayerons d'avoir l'avis de
  professionnels sur ces fonctionnalités et leur implémentation dans les différents
  logiciels. Ensuite on regardera le prix de ces logiciels, verront en quoi cela
  peu être un argument de poids pour les logiciels libres et leur éventuel
  prise de par de marcher. Par la suite nous nous concentrerons sur la documentation,
  livre et autres tutoriels disponible pour ces différents logiciel, et verrons
  quel support est offert aux professionnels pour ces logiciels.

\paragraph{}
  Dans cette étude, nous nous concentrerons sur les plus grande acteur du marché que sont:
  \begin{itemize}
    \item{\textbf{Avid Media Composer:} Leader historique du marché du logiciel de montage non linéaire
      professionnel. Il s'agit du produit phare de Avid Technology publié en 1989 et depuis, ce
      logiciel à joué un rôle essentiel dans l'avènement de ce marché.}
    \item{\textbf{Avid Symphony:} Evolution de Avid Media Composer, il s'agit d'un version plus complète en terme
      de fonctionnalités qui a pour but de répondre aux besoins de monteurs de production longues tel que
      les documentaires et les series télévisées.}
    \item{\textbf{Final cut pro:} logiciel de montage intégré dans la suite de logiciel de post production
          de Apple, Final Cut Studio. Il s'agit d'un logiciel de montage orienté à la fois
          professionnel et création de film. Il est de nos jours très utilisé et est devenu l'un
          des leaders mondiales du marché.}
    \item{\textbf{Adobe Premiere Pro:} Logiciel de montage de la suite Adobe Creative suite, il s'agit du logiciel
      d'édition vidéo à visé professionnel de Adobe System. Il est à la fois adapté pour la création
      de contenu diffusé, mais aussi de contenu post produit}
    \item{\textbf{Cinelerra:} Logiciel de montage libre sponsorisé par la société heroine. Il s'agit d'un logiciel
      de montage non linéaire avec de très nombreuses fonctionnalités. Principalement crée pour le création de contenu
      diffusé, il permet aussi de répondre au besoin de la production de contenu post-produit.}
    \item{\textbf{Kdenlive:} Logiciel de montage libre s'intégrant dans la suite logiciel de l'interface graphique KDE.
      Ce logiciel de montage est assez complet et peu répondre au besoin des monteur de contenu post produit.}
    \item{\textbf{PiTiVi:} Logiciel de montage libre encore basique mais en plein développement. Ce logiciel a pour
      but de répondre au besoin du plus grand nombre, et en particuliers à ceux des professionnels de la création de
      contenu, qu'il soit post produit où non.}
  \end{itemize}

\subsection{Fonctionnalités}
  Tout d'abord, il conviens de voir quel fonctionnalités existent dans les différents
  acteurs du marché.
  \subsection{Prix}
  \subsection{Documentation}
  \subsection{Support}
