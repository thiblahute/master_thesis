\newpage

\subsection{Etat de l'art}

\paragraph{}
La création d'un logiciel open source d'édition vidéo, a été un
objectif pour de nombreux projets logiciels. Du fait de la complexité
inhérente au montage vidéo de manière non linéaire, peu de projets
ont pu durer afin de mûrir et être réellement utilisables par les
utilisateurs comme le montre schéma suivant:

\section{Comparaison des principaux logiciels présents sur le marché de
l'édition vidéo professionnel, et analyse des manques et risques du marché}

\paragraph{}
  Il conviendra d'analyse en profondeur les logiciels existants, qu'ils soient
  propriétaires ou libres. Cette partie a pour but de rendre compte de
  l'état actuel du marcher des logiciels d'édition vidéo qui ont pour principal
  public les professionnelles. Cette étude portant principalement
  sur les logiciels libres, ceux-ci seront évidemment inclus dans cette analyse
  bien que l'on puisse considérer que à cause de leur manque de maturité, ils
  n'y aient pas totalement leur place.

\paragraph{}
  Dans cette optique, on analysera les points clés des logiciels. 
  Tout d'abord on comparera les fonctionnalités des
  logiciels, la manière dont elles sont gérées, et on essayera d'avoir l'avis de
  professionnels sur ces fonctionnalités et leur implémentation dans les différents
  logiciels. Ensuite on regardera le prix de ces logiciels, verra en quoi cela
  peut être un argument de poids pour les logiciels libres et leur éventuelle
  prise de part de marché. Par la suite nous nous concentrerons sur la documentation,
  livres et autres tutoriels disponibles pour ces différents logiciel, et verrons
  quels supports sont offerts aux professionnels pour ces logiciels.

\paragraph{}
  Dans cette étude, nous nous concentrerons sur les plus grands acteurs du marché que sont:
  \begin{itemize}
    \item{\textbf{Avid Media Composer:} Leader historique du marché du logiciel de montage non linéaire
      professionnel. Il s'agit du produit phare de Avid Technology publié en 1989. Depuis, ce
      logiciel a joué un rôle essentiel dans l'avènement de ce marché.}
    \item{\textbf{Avid Symphony:} Evolution de Avid Media Composer, il s'agit d'une version plus complète en terme
      de fonctionnalités qui a pour but de répondre aux besoins des monteurs de productions longues telle que
      les documentaires et les séries télévisées.}
    \item{\textbf{Final cut pro:} logiciel de montage intégré dans la suite de logiciels de post-production
          de Apple, Final Cut Studio. Il s'agit d'un logiciel de montage orienté à la fois
          professionnel et création de film. Il est de nos jours très utilisé et est devenu l'un
          des leaders mondial du marché.}
    \item{\textbf{Adobe Premiere Pro:} Logiciel de montage de la suite Adobe Creative suite, il s'agit du logiciel
      d'édition vidéo à visée professionnelle de Adobe System. Il est à la fois adapté pour la création
      de contenu diffusé, mais aussi de contenu post produit}
    \item{\textbf{Cinelerra:} Logiciel de montage libre sponsorisé par la société Héroïne. Il s'agit d'un logiciel
      de montage non linéaire avec de très nombreuses fonctionnalités. Principalement créé pour le création de contenu
      diffusé, il permet aussi de répondre aux besoins de la production de contenu post-produit.}
    \item{\textbf{Kdenlive:} Logiciel de montage libre s'intégrant dans la suite logicielle de l'interface graphique KDE.
      Ce logiciel de montage est assez complet et peut répondre aux besoins des monteurs de contenu post-produit.}
    \item{\textbf{PiTiVi:} Logiciel de montage libre encore basique mais en plein développement. Ce logiciel a pour
      but de répondre aux besoins du plus grand nombre, et en particulier à ceux des professionnels de la création de
      contenu, qu'il soit post produit où non.}
  \end{itemize}

\subsection{Fonctionnalités}
  Tout d'abord, il convient de voir quelles fonctionnalités existent chez les différents
  acteurs du marché.
  \subsection{Prix}
  \subsection{Documentation}
  \subsection{Support}
