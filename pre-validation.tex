\newpage
\section{Pre-validation}

\subsection{Tasks I am meant to do during the internship}

    \begin {itemize}
        \item {Implement effects in GStreamer editing Service (GES)\ldots
               Done}
        \item {Merge some branches into the Pitivi video editor\ldots
               Some work done here}
        \item {I am not sure what is next, for now the possibilities are:}
        \begin {itemize}
            \item {Work on a video editor for Meego}
            \item {Work on the Pitivi video editor}
            \item {Work on GES (Implementing GESMaterials, and GESRegistry)}
            \item {\ldots}
        \end {itemize}
    \end {itemize}

\subsection{Description of the thesis subject I chose, and aim at getting
            accepted}

  \paragraph{}
      The main idea of my thesis is to write a professional document which can
      be used as a basis for Collabora Ltd to know where we are in the open
      source world concerning the video edition. I want to draw a comparison
      between first, different open source video editors, in particular concerning the
      technical details. And then try to compare them with the ones coming
      from the commercial software world. In this second part, I will be
      comparing them in terms of features and stability since I can't
      find precise details about the technical state of privative softwares.

      I also want to make interviews of the people who can be interested in
      using FOSS for their video editing tasks. I would like to interview
      people working in various fields of the professional video edition,
      in particular:
      \begin{itemize}
        \item {Video clip producers}
        \item {Television content producers}
        \item {Short movie producers}
        \item {Blockbuster producers (also think 3D support)}
      \end{itemize}
      This task will be a big part of my investigation. During the next period
      of the writing of this thesis, I will concentrate on this task. Since
      there is not much documentation about the subject, I will have to create
      most of the content.

      I will also draw a portray of the whole video editing world, which
      means, the different uses that are made of them, the different platforms on which
      they can be used, the  direction it is currently following\ldots

  \paragraph{}
    The idea could probably also be to try to figure out what should be done
    in the open source communities for open source video editing solutions
    to be real alternatives to what exists in the commercial world. This
    would lead to something like:
    \begin {itemize}
      \item{Picture of the video editing world}
      \item{State of the art}
      \item{Analysis of the commercial market}
      \item{What are open source solutions missing?}
    \end{itemize}

\subsection{Relation with my internship}
  \paragraph{}
    This is totally related to my internship since I am actually working on
    open source video editors for several platforms. Collabora is working
    on creating a library called gstreamer-editing-service which offers a
    high level and powerful API to create video editors on top of it. This
    library uses the GStreamer multimedia framework which already offers
    some video edition logics through gnonlin. GStreamer is pretty renowned,
    and is supported by many companies. Collabora exclusively works
    on free and open source software. Its goal being to help companies
    who work with it to get more involved in the open source communities in
    order to get the most of those communities and also to give back to them.
    This way we create efficient open source ecosystems and can ensure the viability
    of it in the long term.

  \paragraph{}
    Collabora is also pushing forward the Pitivi video editor which is
    based on GStreamer. The aim of it is, in the long term, to provide a
    professional video edition framework.

  \paragraph{}
    There is no analysis of the current state of open source video edition, and
    writing this document for my thesis seems something pretty interesting
    for me as well as for Collabora Multimedia.

\subsection{Possible subject formalization}
  \paragraph{}
    The exact subject of my thesis is not define precisely yet, but there are several
    questions that have already come to my mind:
    \begin{itemize}
      \item {Open source video editor: are they ready for professional use?}
      \item {Are open source video editors good alternatives to the ones
             coming from the commercial world for professional use?}
      \item {What is the future of Open Source video editors as far as
             professional editors are concerned?}
      \item {Open source video editors: what is missing for them to be
             alternatives for professional use?}
    \end{itemize}

\subsection{Point of view}
  \paragraph{}
    I will write this document from an open source 'proselyte' point of
    view. Being as neutral as possible, this document will be focused on
    analyzing the market, and actually analyze the open source solutions from
    a technician point of view. I will keep in mind that the current state
    of open source video editor is far from being perfect. But I do believe
    that we are going in the good direction. I will focus a lot on the future,
    and what is going on currently, in order to have a vision of the future,
    writing about the state of the art first in order to make a retrospective 
    analysis of the subject.
