\newpage
\section{Fonctionnalités clefs de tout éditeur vidéo à visée
  professionnelle}

\paragraph{}
  Le montage vidéo professionnel est un domaine très vaste, et l'on peu
  s'attendre à ce que les besoins auxquels doivent répondre les logiciels
  permettant de produire les différents types d'œuvres audiovisuel
  varient fortement en fonction du type de production audiovisuel. Nous nous
  devons donc de classer les différentes productions audiovisuel par
  catégories. Puis ensuite il sera possible d'analyser les besoins en
  terme de logiciel de montage requis afin de produire ces document. Pour finir
  nous verrons si les besoins sont variés, et essayerons de trouver les
  fonctionnalités qui sont communément utilisées afin de produire ces œuvres.

  \subsection{Définition du marché}
    Dans un premier temps, nous allons définir et analyser les différentes type
    de productions audiovisuels professionnelles.
    \subsubsection {Les court métrages}
      La production de audiovisuel a commencer par le court métrage, tout d'abord ils
      étaient en noire et blanc, et sans son, puis il se sont perfectionné par la suite,
      jusqu'à arriver à ce que l'on appel film de nos jours.
      A l'heure actuel, les court métrages sont utilisé pour faire passer des messages,
      ou exposer des idées de manière plus brève et intense. De par leur caractères
      cours, il est intéressant de se poser la question de savoir si dans ce genre
      d'œuvre, les monteurs utilisent des techniques spéciales qui permettraient
      de les rendre plus dynamique et si des fonctionnalités spécial sont utilisé dans
      ce but.  %TODO
    \subsubsection {Les films}
      Le film est un monde à part, il est divers et varier, et dépendant du style de film,
      on peu se demander si les monteurs, n'utilise pas des techniques, fonctionnalités
      différentes. Il n'est pas ``relevant'' de décrire les différent type de film ici,
      mais il serait intéressant de constater les différences que au montage qu'implique
      le genre du film.

      Il n'a pas été possible d'interviewer de monteur de film jusqu'à maintenant, mais
      le livre ``The technique of film and video editing, History, Theory, and Practice''
      \cite{TheTechniqueOfFilmAndVideoEditing} est un bon point de départ pour
      comprendre le montage de cinématographique et la très grande influence qu'il a
      sur les autres type de productions audiovisuels.
      %TODO essayer de trouver des monteurs de films!
    \subsubsection {Les séries télés}
      Au niveau du montage, ce type de production n'a à priori pas caractère spécial, il
      s'agit plus ou moins d'œuvres longues découper en plusieurs épisodes.
    \subsubsection {Les publicités}
      La publicité a quelque chose de similaire avec les cours métrage au niveau du
      montage dans le sens où il s'agit de création très courte et généralement dynamique
      mais bien évidemment la visée, est totalement différente. Le but de ces productions
      est d'attirer des consommateurs, et de ce fait, les monteurs utilisent des
      techniques spéciales.
    \subsubsection {Les documentaires}
      %TODO
    \subsubsection {Les émission télévisés (journaux télévisés...)}
      %TODO
    \subsubsection {Les clips vidéos}
      Le clip vidéo est un contenu visuel synchronisé avec la musique, on peu se
      rendre compte que dans ce type de vidéos, il y a souvent beaucoup d'effets
      spéciaux, et demande à priori une très grande précision au niveau de la
      synchronization entre le son et l'image. %TODO

  Afin de définir les besoins, nous avons interviewer différents
  monteurs professionnels annexes 1 en essayant de couvrir le %FIXME
  maximum de champs de l'édition vidéo. Nous avons pu récolter des
  informations provenant de monteurs de Clips vidéos, de courts métrage,
  de publicités, de reportages, . Ils est apparue intéressant que dans %FIXME
  les fait, les fonctionnalités utilisés sont assez similaire bien que
  les œuvres finales soient totalement différentes. Dans cette partie,
  il sera aussi donner de comprendre


  \subsection{Fonctionnalité communes}
  \subsection{Fonctionnalité spécifiques}
