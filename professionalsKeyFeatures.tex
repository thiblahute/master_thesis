\newpage
\section{Fonctionnalités clefs de tout éditeur vidéo à visée
  professionnelle}

\paragraph{}
  Le montage vidéo professionnel est un domaine très vaste, et l'on peu
  s'attendre à ce que les besoins auxquels doivent répondre les logiciels
  permettant de produire les différents types d'œuvres audiovisuel
  varient fortement en fonction du type d'œuvre. Il s'agit dans les fait
  de différents cas d'utilisation d'un même logiciel, et afin d'étudier
  les possibilités des logiciels libres dans ce domaine, il faut
  définir ces use case, les fonctionnalités qui en découlent\ldots
  Nous nous allons définir les principaux en cas d'utilisation en fonction des
  différents types de productions audiovisuel et ainsi définir les fonctionnalités
  nécessaires pour répondre à ces cas d'utilisation.

  Ensuite on analysera le plus petit dénominateur commun entre les
  fonctionnalités que doivent avoir les logiciels de montage afin de répondre
  aux besoins des différent producteurs audiovisuels. Pour finir
  nous verrons si les besoins sont variés, et essayerons de trouver les
  fonctionnalités qui sont propres à chaque type de production. Cette première
  analyse a pour but de clarifier les besoins des professionnels afin de
  déterminer par la suite quels sont ceux auxquels on répond dors et déjà %How the hell do u write that!
  lesquels on peu prétendre répondre dans un future proche, et lesquels
  son hors du scope actuel des technologies libre.
  %FIXME Check if it is the place to say that...

  \subsection{Les bases de l'édition vidéo}
    \paragraph{}
      Tout d'abord, il est évident que pour qu'un logiciel de montage puisse répondre
      aux besoin de professionnel, les fonctionnalités basiques de l'édition vidéo
      doivent être couvertes, cette partie a pour but de définir quels sont ces
      fonctionnalités, et les expliquer succinctement:
      %TODO Look for a def of non-linear video editors

    \subsubsection{Gestion des footage}
      Un logiciel d'édition vidéo doit permettre d'importer les footages a partir
      desquels on veut faire le montage, c'est à dire les fichiers vidéos, audios,
      et images avec lesquels on travail. Il doit être possible de prévisualiser ces
      clips.

    \subsubsection{Gestion de la timeline}
      La timeline, est la partie de l'interface dans laquelle on va disposer les
      différents clips. Certaines fonctionnalités sont indispensables en ce qui
      concerne cette partie du logiciel:
        \begin{itemize}
          \item{Découpages des clips}
          \item{Unlinking de la piste audio et de la piste vidéo}
          \item{Gestion des in point et  out point des clips} %Should that be translated?
          \item{Notion de layer}
          \item{Mixing des layer}
        \end{itemize}

      \paragraph{}
        Il s'agit là des fonctionnalités très basiques de l'édition vidéo auquel
        aucun logiciel de montage ne peut échapper.

  \subsection{Définition du marché}
    \paragraph{}
    Dans un premier temps, nous allons définir et analyser les différentes types
    de productions audiovisuels professionnelles, et les besoins en termes de
    fonctionnalités dont les monteurs ont besoins.


    \paragraph{}
    Afin de définir les besoins, nous avons interviewer différents
    monteurs professionnels (annexes 1) en essayant de couvrir le %FIXME
    maximum de champs de l'édition vidéo. Nous avons pu récolter des
    informations provenant de monteurs de clips vidéos, de courts métrage,
    de publicités, de reportages. %TODO

    \subsubsection{Les court métrages}
      \paragraph{}
      La production audiovisuel a commencer par le court métrage, tout d'abord ils
      étaient en noire et blanc, et sans son, puis il se sont perfectionné par la suite,
      jusqu'à arriver à ce qu'il est aujourd'hui.

      \paragraph{}
      A l'heure actuel, les courts métrages sont utilisés pour faire passer des messages,
      ou exposer des idées de manière brève et intense. De par leur caractères
      cours, il est intéressant de se poser la question de savoir si dans ce genre
      d'œuvre, les monteurs utilisent des techniques spéciales qui permettraient
      de les rendre plus dynamique et si des fonctionnalités spécial sont utilisé dans
      ce but.
      \paragraph{}
        Dans la production de ce type d'œuvre, les fonctionnalités qui sont indispensable
        selon les interviews sont:
        \begin{itemize}
          \item{Transition (fading en priorité)}
          \item{Effets basiques tel que le passage en noire et blanc\ldots}
          \item{Time remmaping}
          \item{Retouche des couleurs}
        \end{itemize}

    \subsubsection {Les publicités}
      \paragraph{}
        La publicité a quelque chose de similaire avec les cours métrage au niveau du
        montage dans le sens où il s'agit de création très courte et généralement dynamique
        mais bien évidemment la visée, est totalement différente. Le but de ces productions
        est d'attirer des consommateurs, et de ce fait, les monteurs utilisent des
        techniques spéciales mais les fonctionnalités du logiciel nécessaires restent
        principalement:
          \begin{itemize}
            \item{Transition (fading en priorité)}
            \item{Effets basiques tel que le passage en noire et blanc\ldots}
            \item{Time remmaping}
            \item{Retouche des couleurs}
            \item{Création et ajout de génériques}
          \end{itemize}

      \paragraph{}
        La qualité du rendu étant très importante, des logiciels spécialisé sont utilisés
        afin de crée le contenu (Audio, effets, images\ldots). Mais il a apparait que dans
        la création de publicité, il arrive fréquemment que le contenu soit monter et
        éditer en très grande partie directement dans le logiciel de montage.

    \subsubsection {Les films}
      \paragraph{}
        Le film est un monde à part, les moyens étant généralement très élevés et la
        qualité du rendu étant très soigneusement géré, les méthodes utilisés
        pour la post production sont assez différentes de ce que l'on trouve dans la
        création d'autres types de contenu.

      \paragraph{}
        Il n'a pas été possible d'interviewer de monteur de film jusqu'à maintenant, mais
        le livre ``The technique of film and video editing, History, Theory, and Practice''
        \cite{TheTechniqueOfFilmAndVideoEditing} est un bon point de départ pour
        comprendre le montage de cinématographique et la très grande influence qu'il a
        sur les autres type de productions audiovisuels. On peu considérer le film comme
        étant l'œuvre audiovisuel par excellence. %FIXME formulation pourri

      \paragraph{}
        Dans le monde du cinéma, le logiciel de montage vidéo est l'un des logiciel
        parmi un système connecté de logiciel de post production. Des spécialistes de
        différents domaines crée les parties du film, et le monteur a pour but
        de lier tout ces éléments au travers du logiciel de montage. Les logiciels
        de post production sont entre autre:
        \begin{itemize}
          \item{Éditeur de son}
          \item{Création d'effet}
          \item{Retouche d'image}
          \item{Création d'animation}
          \item{\ldots}
        \end{itemize}

      \paragraph{}
        Ce qui résulte dans le fait que le logiciel de montage vidéo à proprement parlé ne
        demande pas vraiment de fonctionnalités très évolue, la base de l'édition
        et la possibilité d'organiser l'immense quantité de footage de manière efficace
        semblent être les seuls éléments clefs dans ce domaine. Les autres logiciels de
        post production sont bien évidemment aussi nécessaire afin de permettre de faire
        le montage de films, mais cela est un élément auquel ce document n'est pas destiné
        à répondre dans le détail.

      \paragraph{}
        Une autre caractéristique de la production cinématographique, qui découle une
        fois de plus du fait que la qualité du résultat doit être irréprochable, est
        que les logiciels de montage doivent permettre de visualiser chaque image du
        film de manière très précise (le montage de film ce fait dans certain cas en
        choisissant chaque image depuis un tableau de frames).

        %FIXME, faudrait plus détailler ici?

      \paragraph{}
        Bien que ne demandant pas vraiment de fonctionnalité très avancer, la création
        de film a des besoins assez évoluer en ce qui concerne le logiciel de montage:
        \begin{itemize}
          \item{Organisation très avancer des footages}
          \item{Création et ajout de générique}
          \item{Passerelles avec le reste des logiciels de post production}
          \item{Preview de chaque frame dans le détail}
        \end{itemize}

      %TODO essayer de trouver des monteurs de films!

    \subsubsection {Les séries télévisés}
      Le niveau de qualité des séries télévisés n'étant aussi élevé que pour
      le montage des films, les traitements sont la majorité du temps réalisé
      directement dans le logiciel de montage même. Cela implique un nombre de
      fonctionnalité plus important avec comme nécessité:
      \begin{itemize}
        \item{Création et ajout de titre}
        \item{Création et ajout de génériques}
        \item{Retouche des couleurs}
      \end{itemize}

    \subsubsection {Les documentaires}
      %%TODO!!
      Le documentaire est en général assez sobre en terme de montage, il est en
      fait dans pour la plupart dans le logiciel de montage, mais ne demande pas
      de fonctionnalité spécial. En général, les fonctionnalités utilisé
      pour produire ce type d'œuvre sont:
      \begin{itemize}
        \item{Création et ajout de titre}
        \item{Création et ajout de génériques}
        \item{Retouche des couleurs}
        \item{Transition smpte\footnote{FIXME smpte}}
      \end{itemize}

    \subsubsection {Les émission télévisés (journaux télévisés...)}
      Les émissions de télévisions sont aussi assez spécial dans le monde du montage
      du fait que la plupart du temps le montage doit être réaliser très rapidement,
      voir en direct, et par conséquent, les logiciels de montages doivent être
      adapté cette contrainte. Par conséquent, les fonctionnalités utilisé pour
      ce type de création sont:
      \begin{itemize}
        \item{Fonctionnalité de template qui permet la mise en place des titres,
              présentations, au moment voulu et ainsi faire le montage en direct}
        \item{Titres}
      \end{itemize}

    \subsubsection {Les clips vidéos}
      Le clip vidéo est un contenu visuel qui a pour but de représenter
      une certaine musique, on peu se rendre compte que dans ce type de
      vidéos, il y a souvent beaucoup d'effets spéciaux, et demande à
      priori une très grande précision au niveau de la synchronisation
      entre le son et l'image. La track audio dans de tel production
      sera de préférence effectué avec un logiciel dédié à cette
      effet. Pour résumer, les fonctionnalités nécessaires sont:
      \begin{itemize}
        \item{Création de titres complexes (Titre en mouvement, etc\ldots)}
        \item{Ajout de titres}
        \item{Ajout d'effets}
        \item{Utilisation avancé des keyframes}
        \item{Time remapping}
      \end{itemize}

  \subsection{Analyse des fonctionnalités communes}
    On s'aperçoit donc que de nombreuses fonctionnalités sont commune au différents
    types d'œuvre. Il convient de détailler chacune de ces fonctionnalité afin de
    nous rendre compte de ce qu'elles implique en terme de logiciel de montage.

    \subsubsection{Création et ajout de titre}
      \paragraph{}
        Cette fonctionnalité est utilisé dans la création de plusieurs type de contenu:
      \begin{itemize}
        \item {Série télévisés}
        \item {Documentaires}
        \item {Clips vidéos}
      \end{itemize}
      \paragraph{}
        Bien que cette fonctionnalité soit utilisé dans ces différents types de contenu,
        ce qu'elle implique dans le logiciel à proprement parler peu varie en fonction
        de différent paramètre. Par example, dans une série télévisés en général le
        travail sur les titre sera assez limité, on aura en général une vidéo en arrière
        plan et un titre que fera un fondu arrière. Alors que dans le cadre de clips vidéo,
        il sera fréquent que le titre soit en mouvement et qu'il suive le rythme de la
        musique par example. Afin de répondre au besoin du plus grand nombre, il faudrait
        pouvoir répondre à ces différent cas d'utilisation, mais il sera plus difficile
        aussi bien en terme de backend qu'en terme d'interface Utilisateur de répondre
        aux besoins le plus avancés.

    \subsubsection{Création et ajout de générique}
      \paragraph{}
        La création de générique est une fonctionnalité auquel de
        nombreux monteur en particulier professionnels font appel. Il est
        indispensable  pour la création de contenu professionnel, de pouvoir
        créer et mettre en place un générique dans la vidéo. Cette
        fonctionnalités en terme de backend est similaire à
        celle des titres puisqu'il s'agit ni plus ni moins d'ajouter du
        texte au dessus d'un fond qu'il soi animé ou non. Mais en terme d'UI
        \footnote{ UI: User Interface, il s'agit du terme  très largement
          employé pour définir l'interface utilisateur, en général graphique (GUI)},
        il s'agit belle et bien de deux fonctionnalités différentes puisque de par
        définition, le générique est un texte défilent, qui va, dans une très grande
        majorité des cas, de haut en bas.

      \paragraph{}
        Cette fonctionnalité est l'une des plus basiques si
        l'on veut pouvoir répondre aux besoins des professionnels. Elle est utilisés
        dans la plupart des créations vidéo et doit être à priori standardiser et
        simple à utiliser dans l'interface utilisateurs puisque les monteurs ne devrai
        pas perdre beaucoup de temps à mettre en place le générique (le contenu étant
        créé par d'autres personnes dans la majorité des cas).

  \subsection{Fonctionnalité spécifiques}
    Ils est apparue intéressant que dans %FIXME
    les fait, les fonctionnalités utilisés sont assez similaire bien que
    les œuvres finales soient totalement différentes.

    \paragraph{}
    Quelques fonctionnalités sont apparue comme vraiment propre à la création
    d'un type d'oeuvre en particulier.

    \subsubsection{Visualisation image par image:}
      Dans le cadre de la création de film, la preview de chaque frame,
      de manière précise semble être une fonctionnalité essentiel,
      cela signifie, que le logiciel de montage doit avoir un moyen de
      voir de manière simple chaque frame des vidéo présente dans la
      timeline. Cette fonctionnalités est aussi utile dans le cadre de
      la création d'autres oeuvre, mais est indispensable dans le cadre
      de film, afin de s'assurer de la qualité du résultat. En effet,
      lors de la création d'un film, chaque frame doit être contrôlé,
      alors que dans d'autre types d'oeuvre, les exigences étant moins
      élevés ainsi que les moyens, une telle fonctionnalité ne est pas
      indispensable.

    \subsubsection{Gestion avancé des Keyframes:}
      Dans la création de clips en particulier, 

  \paragraph{}
    On constate donc que le champs de fonctionnalités est vaste, de plus dans ce
    document, on ne s'intéresse qu'aux principales fonctionnalités, mais de chacune
    d'entre elle, peuvent découler de nombreuse autres. Par exemple, dans le cadre
    de la création de générique, les monteurs peuvent vouloir mêler à la fois de la
    vidéo et du contenu textuel défilent, avec des transition sur la vidéo et sur le
    texte de manière décaler, cela
