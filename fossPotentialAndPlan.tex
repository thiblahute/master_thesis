\chapter{Analyse des opportunités des technologies libres dans le
domaine de l'édition vidéo et prévisions}

\minitoc \newpage

\paragraph{}

Maintenant que les besoins et que les solutions existantes ont été
analysées on rendra compte de la situation actuelle des technologies
libres et de leurs communautés. Il est aussi important de chercher les
raisons qui expliquent que ces logiciels ne sont pas utilisés par les
professionnels. Puis, nous essayerons d'envisager les solutions possibles
qui permettraient de remédier à cette situation.

\paragraph{}

Dans cette partie, nous analyserons la différence entre les manières
d'envisager la création de logiciel et nous verrons quels sont les
avantages et inconvénients de ces fonctionnements. Par la suite nous
nous concentrerons sur les frameworks existants pour faire une analyse
technique des ces technologies. Par la suite, nous analyserons les
communautés qui portent ces différents projets afin d'arriver à voir
les lacunes et les avantages de chacun des projets.  Pour finir, nous
tirerons les conclusions de cette analyse afin de trouver des solutions
aux défis qu'est la création d'un logiciel libre de montage vidéo.

\newpage

\section {Etat actuel de l'offre de logiciel libre}

Le schéma suivant permet de résumer facilement la situation:

\begin{figure} [h]
  \begin{center}
    \includegraphics[width=0.9\textwidth]{images/open-source-video-editor-timeline}
  \end{center} \caption{Open source video editors timeline (Auteur:
  Jean-François Fortin, PiTiVi designer)} \label{Yes}
\end{figure}

\paragraph{ }

On constate donc que de nombreux projets de logiciel libre de montage
vidéo on vu le jours ces 10 dernières années, ayant différents
objectifs.  On peu distinguer deux types de public visés par ces
projets:

\begin {itemize}

  \item {Les amateurs}

  \item {Les professionnels ou semi-professionnels}
\end {itemize}

\paragraph {Les amateurs de montages vidéo}

\subparagraph{}

Plusieurs projets libres permettent ou visent à répondre aux besoins
des amateurs, mais à l'heure actuelle même ce cas d'utilisation n'est
pas pleinement satisfait par les logiciels libres. Parmi les logiciels
dont l'objectif est de permettre de créer des montages simples on distingue:

\begin {itemize}

  \item {openshot: Logiciel présentant  de nombreuses fonctionnalités, mais dont la
    qualité d'implémentation présente des faiblesses.}

  \item {kino: Logiciel avec un nombre de fonctionnalités limité permettant de faire des
    petit montages éfficacement}

  \item {Vidiot qui vise la production de vidéo amateur simple}

\end {itemize}

\paragraph {}

Mais les logiciels ayant pour objectif de pourvoir aux besoins plus
avancés en particulier à ceux des professionnels (précédemment présenté
dans le cadre de la définition des plus grands acteur du marché)
peuvent être utilisés dans le cadre de montage amateur.

\paragraph{}

Un nouveau projet a aussi récemment vu le jour, dont la finalité est assez
différente des logiciels actuellement présents. Il s'agit de Novacut,
qui permet aux créateurs de films et séries web de
faire le montage de manière collaborative à travers d'Internet, en
partageant les ressources (footage).

\paragraph{}

En définitive aucun projet n'a encore réussi à s'imposer et ainsi
regrouper les développeurs au sein de projets majeurs. Dans d'autre
domaines, cela a été le cas, par exemple dans le domaine des lecteurs
vidéo, Vlc a su surpasser ses concurrents, et ainsi supplanter le
marché des lecteurs vidéo, qu'il soit libre ou non. Dans le domaine
des environnements de Bureau graphique, KDE et Gnome sont arrivés à un
stade où leur supériorité technique, et en terme de fonctionnalités,
fait d'eux des plateformes de référence.

\paragraph{}

Il est donc intéressant de se demander quelles technologies et quels
logiciel(s), pourraient se voir attribuer cette place dans le monde
de l'édition vidéo libre. Nous allons donc analyser les logiciels
et les technologies libres les plus avancés, (précédemment mentionnés
dans le cadre de l'analyse de marché: Cinelerra, Kdenlive et PiTiVi).
Nous verrons ainsi s'ils ont le potentiel de pouvoir un
jour rivaliser avec les logiciels propriétaires sur le marché très
fermé du montage vidéo professionnel.

\paragraph{}

NB: Il aurait été intéressant d'analyser le logiciel lightworks,
en voie de libération, mais à l'heure actuelle, aucun code n'a été
libéré, et par conséquent, celui-ci ne peut pas faire partie de cette
analyse.

\newpage

\section{Technologies}

\paragraph{}

Pour faire une analyse technique des produits permettant de faire
de l'édition vidéo, il est nécessaire d'analyser le ``core'' des
logiciels, c'est à dire la partie du logiciel où les opérations
d'édition sont effectivement réalisées. Dans ces domaines, il existe
deux façon de procéder:

\begin{itemize} \setlength{\itemsep}{2mm}

  \item{Création d'un logiciel monolithique\index{monolithique}}

  \item{Création d'un framework \glossary {name={framework},
   description={ Un framework est un ensemble d'outils et de composants
   logiciels organisés conformément à un plan d'architecture et des
   design patterns}} \index{framework}}

\end{itemize}

\subsection {Technologies monolithiques\index{monolithique} VS technologies
modulaires, frameworks}


\subsubsection{Logiciels monolithiques \index{monolithique}} %FIXME Look
                                                             %for a def

\paragraph{}

Le conception monolithique \index{monolithique} dans le cadre des
logiciels d'édition vidéo, consiste à développer au sein d'un même
entité de code:

\begin{itemize} \setlength{\itemsep}{2mm}

  \item {la partie graphique et la partie de calculs
    permettant la gestion de tout ce que l'édition non linéaire
    implique}

  \item {L'interface utilisateur.}

\end {itemize}

\paragraph{}

Par le terme logiciel monolithique\index{monolithique}, il faut
réaliser que le logiciel peut utiliser des librairies externes,
mais le core de ce même logiciel, et la logique d'édition linéaire
à proprement parler sont directement faits à l'intérieur du logiciel
et non par une librairie ou framework \index{framework} externe. Cela a
pour principal avantage de présenter une conception simplifiée pour les
raisons suivantes:

\paragraph{}

Les logiciels professionnels (commerciaux) utilisent très probablement
tout ce mode de fonctionnement (même si probablement, en interne il ont
un core qui ressemble fortement à un framework \index{framework}).dans le
monde des logiciels libres,les développeures de Cinelerra ont décidé
d'utiliser ce mode de fonctionnement.

On peut voir plusieurs conséquences immédiates de ce mode de
développement:

\begin{itemize} \setlength{\itemsep}{2mm}

  \item {Les développeurs n'ont pas la nécessité de penser
    en terme d'interface publique de programmation (API\index{API}), et
    n'ont pas à garantir la stabilité de celle-ci: le risque réside dans le fait que
    la qualité de l'architecture ne soit pas optimale car
    la création d'API\index{API} oblige les développeurs/architectes
    à réellement analyser les besoins de manière plus large dès le
    début de la conception. Dans le cas où l'on ne crée pas d'interface
    publique de programmation vouée à être réutilisée, le risque est
    que le travail de design et d'architecture ne soit pas réalisé,
    et que le code grandisse de manière anarchique avec les différents
    développeurs qui font des extensions au fur et à mesure de leurs besoins.}

  \item {Les développeurs n'ont besoin de penser l'architecture
	seulement pour les cas d'utilisation qui sont liés à ce même logiciel:
    ils n'ont pas à voir au delà de ces use cases.}

  \item {Les erreurs en terme de design n'ont pas d'incidences aussi
    graves que dans le cas d'un framework\index{framework}.}
\end {itemize}

\paragraph{}

On se rend compte que cette manière de faire a pour principal avantage
le fait que le logiciel peut être développé plus rapidement puisque
le core du logiciel, et donc le code qui implémente la logique de
l'édition non linéaire, est conçue avec pour seul cas d'utilisation,
celui du logiciel. Cependant, de nombreux inconvénients existent à cause de
la nature monolithique\index{monolithique} du design:

\subparagraph{Besoins en main d'oeuvre considérables:}

\subparagraph { }

Dans le cadre de logiciel d'édition vidéo, le code à produire est
considérable, comme le montre les statistiques (Annexes 2). Le logiciel
Cinelerra à lui seul fait plus d'un million de lignes. Une telle
quantité de code est difficile à maintenir et requiert des ressources
importantes en terme de main d'oeuvre. Le fait que le logiciel soit
monolithique\index{monolithique} implique que celui-ci va être utilisé
seulement par ce logiciel, et par conséquent, les développeurs ne peuvent
pas compter sur d'autre utilisation de ce code pour améliorer et développer
le core du logiciel.

\paragraph{Réutilisabilité:}

\subparagraph { }

L'un des inconvénients de cette manière de faire est que le code que
l'on a à l'intérieur du logiciel n'est pas réutilisable directement
par d'autres projets, et par conséquent, on peut considérer que cela
est ``individualiste``, chose qu'il convient d'éviter dans le cadre du
développement de logiciel libre afin de ne pas multiplier les efforts,
et dupliquer le code.

\paragraph{}

Cette façon de faire a été utilisée par le projet Cinelerra. Ce
logiciel est le plus avancé en terme de fonctionnalités que le
marché des logiciels libres de montage offre. On peut penser que son
architecture monolithique\index{monolithique} explique ce développement
plus abouti, bien qu'il y ait évidemment de nombreux autres facteurs qui
interviennent en particulier le fait que ce logiciel a été développé par
la société Heroine Virtual.

\subsubsection {Utilisation de  frameworks \index{framework}}

\paragraph{}

L'autre possibilité est de séparer en deux parties
bien distinctes l'implémentation de la logique de l'édition, lecture,
encoding vidéo (core logiciel), de la partie graphique, interaction
avec l'utilisateur final.

\paragraph {Le framework}

\subparagraph{}

La grande différence entre la conception monolithiques
\index{monolithique} et la création d'un framework \index{framework}
réside dans le le fait que dans le cadre d'un framework, on développe
une API \index{API} autour du core du logiciel. Cela résulte dans le
fait que le core est un programme (librairie) externe, réutilisable par
n'importe quel autre application.  On peut considérer que les avantages
des frameworks sont les inconvénients des applications monolithiques
\index{monolithique} et vice-versa. L'avantage principal des frameworks sur
une conception monolithique\index{monolithique} est la possibilité de
partager un même code à travers de multiples applications. Celà permet de
réunir les efforts au travers, dans notre cas précis, de tout type
d'application multimedia.

Dans le cadre de l'édition vidéo, on peut encore distinguer deux manière
d'envisager son développement:

\begin {itemize}

  \item {Utiliser un framework multimedia généraliste, et créer les
  outils nécessaire
         au montage au dessus de celui-ci} %stupid french!\ldots On top
                                           %of it?

  \item {Créer un framework spécialement orienté montage vidéo}

\end {itemize}

Dans le monde du logiciel libre, ces deux manières d'envisager le
développement d'un framework multimedia ont été abordées par les deux
projets de framework leader sur ce segment:

\begin {itemize}

  \item {MLT qui se définit comme étant un ``Framework multimedia design
    et développé pour le brodcasting télévisé.''}

  \item {Gstreamer qui se définit comme étant un ``framework multimédia
    basé sur la notion de pipeline ce qui lui permet de nombreux types
    d'applications multimedia tels que des lecteurs multimédia, des
    logiciels de broadcasting, des logiciel de montage vidéo\ldots''}

\end {itemize}

\subparagraph {}

Au dessus de ces frameworks, deux applications (interfaces graphique)
de montage vidéo se sont développées.

\begin {itemize}

  \item {PiTiVi: utilise le Framework multimedia GStreamer}

  \item {Kdenlive utilise le framework\index{framework} orienté édition et
    broadcasting MLT.}

\end {itemize}

\paragraph {}

Dans le cadre des Frameworks, nous nous intéresserons
en particulier à l'analyse de ceux-ci puisque les notions relatives
à l'édition vidéo, et la gestion de toute la partie multimédia est
réalisée par ceux-ci. Les logiciels d'édition ne sont à priori que
de simples interfaces graphiques basées sur ces frameworks, et par
conséquent leur analyse ne présente qu'un faible intérêt.

\newpage \section{Analyse technique}

\paragraph {}

Dans cette partie nous allons analyser les entrailles des trois logiciels précédemment
définis: Cinelerra, PiTiVi et Kdenlive.

\subsection{Cinelerra:}

\paragraph {}

Cinelerra est le logiciel d'édition audio et vidéo et de composition le plus
avancé dans le monde de logiciel libre. Il est développé principalement par
Adam Williams pour l'entreprise Heroine Virtual Ltd. Ce projet a été initié
en 2001 sous le nom de broadcast2000 par cette même entreprise.

\paragraph{}

Le logiciel Cinelerra a été développé principalement en C++ et est composé de 3 parties
principales interdépendantes:

\begin{itemize}

  \item{Lecture/et rendering audio vidéo}

  \item{Edition vidéo non-linéaire}

  \item{Interface graphique}

\end{itemize}

\subsubsection{Lecture et rendering}

\paragraph{}

Dans le cadre de la lecture audio et video, Cinelerra fait appelle à diverse library:

\begin{itemize}

  \item{ffmpeg: Solution compete, cross plateforme d'enregistrement, lecture, conversion
    de flux audio et vidéo. Il inclue libavcodec, librairie leader dans le domaines des
    codec.
    Il s'agit du core de la lecture audio et vidéo de Cinelerra.}

  \item{faac/faad: AAC audio encoder}

  \item{x264: h264 encoder}

  \item{x264: h264 encoder}

  \item{\ldots}

\end{itemize}

\subparagraph{}

Toutes ces librairies sont utilisées dans le but de lire et écrire des fichiers multimedia.
Afin de standardiser, et permettre l'utilisation de ces libraries de manière similaire
au sein du logiciel, les développeurs de Cinelerra ont élaboré au cas par cas des pont entre
ces librairies et le reste du logiciel (Fichier dans le dossier quicktime).

\paragraph{Edition non linéaire}


\newpage \section{Analyse des communautés}

\newpage \section{Lacunes}

\newpage \section{Solutions possibles}
