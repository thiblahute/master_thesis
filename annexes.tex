\newpage
\chapter*{Annexes}
\subsection {Interview de Sophian Veri,
Jeune entrepreneur dans le milieu de l'édition vidéo}
%TODO meilleur mise en page

\paragraph{}
1-  Quel logiciel d'édition vidéo utilisez vous à l'heure actuel?
Sophian Veri: J'utilise exclusivement finalcut

\paragraph{}
2- Quel types de vidéo produisez vous?
Sophian Veri: Je produis des clips, reportages, pubs et courts métrages

\paragraph{}
3- Ce logiciel répond t il à tout vos besoin en terme de montage?
Sophian Veri: généralement oui

\paragraph{}
4- Quels défauts vous viennent à l'esprit quand vous penser à cette outil?
Sophian Veri: le montage multicamera mal géré, temps de rendu trop long car le
logiciel ne fonctionne pas avec toute la mémoire vive de l'ordinateur
contrairement a la suite adobe

\paragraph{}
5- Considérez vous finalcut comme étant la meilleur solution de montage,
si oui, pourquoi?
Sophian Veri: Oui, sans aucun doute. Pour sa facilité d'utilisation,
les codecs accepté y sont nombreux et la liste des effets est
longue\ldots

\paragraph{}
6-  Quels fonctionnalité utilisez vous au quotidien (ex: Multicamera, effets,
transitions, keyframes, time remapping, outils collaboratifs, proxy
editing, templates...)

Sophian Veri: effet de mauvais téléviseur, time remap, cache patate (qui m'évite
de passé par after.) ralenti, fondu enchainé
Thibault Saunier: Qu'es que cache patate?
Sophian Veri: cache patate, c'est masque dans after

\paragraph{}
7-  Quels fonctionnalités considérez vous comme indispensable, (même si vous
ne les utilisé pas au quotidien)?
Thibault Saunier: Oui je vois. Et il y en a d'autre dont vous avez forcement besoin?
Sophian Veri: la modification des couleurs  tout ce qui es travail de
l'image contraste netteté saturation

\paragraph{}
8- Quel pourcentage des fonctionnalité du logiciel pensez vous utilisez
en tout?
Sophian Veri: il y a  tellement de fonctionnalités que je serai tenté
de dire 40%

\paragraph{}
9- Seriez vous prêts à utilisez des logiciels ayant moins de fonctionnalités,
mais qui répondrais de manière plus efficace à vos besoins?
Sophian Veri: pourquoi pas a condition qu'il soi aussi intuitifs et que les effets
que j'utilise soient tout aussi bien gérés

\paragraph{}
10-  Le prix du logiciel est-il un critère de choix selon vous?
Sophian Veri: oui, en tant que nouvelle jeune entreprise, le prix est
un critère de choix

\paragraph{}
11- Avez-vous des problèmes de stabilité (de bugs) avec finalcut?
Sophian Veri: les bugs, assez rarement

\paragraph{}
12- La dépendance vis-à-vis du créateur du logiciel que vous utilisez
vous parait il être quelque chose de dangereux?
Sophian Veri: oui dans le sens ou on ne sait jamais quels transformations le
logiciel subira avec la version suivante et il se peu que la version devienne
moins bien et qu'elle ne me satisfasse pas de la même manière. Comme avec
finalcut 10 qui a l'air d'être raté.

\paragraph{}
13- Connaissez vous certains logiciels libre d'édition vidéo?
Sophian Veri: Je ne sais pas vraiment ce que c'est.

Thibault Saunier: Merci bien d'avoir pris le temps de répondre
Sophian Veri: mais de rien
...

\subsection {Interview de Karim Hachemi, Monteur chez Falfyprod}

\paragraph{}
1-  Quel logiciel d'édition vidéo utilisez vous à l'heure actuel?
Sophian Veri: Final cut mais surtout Adobe CS5

\paragraph{}
2- Quel types de vidéo produisez vous?
Sophian Veri: clip, pub, institutionnel

\paragraph{}
3- Ce logiciel répond t il à tout vos besoin en terme de montage?
Sophian Veri: Adobe Première non mais final cut oui le problème est que les
rush du 5D sont mal accepter par final cut donc besoin de les
convertir donc perte de temps enorme

\paragraph{}
4- Quels défauts vous viennent à l'esprit quand vous penser à cette outil?
Sophian Veri: Final cut est un tout petit peu plus compliqué mais a
est plu complet enfin c'est e que j'ai comme impression


\paragraph{}
5-  Quels fonctionnalité utilisez vous au quotidien (ex: Multicamera, effets,
transitions, keyframes, time remapping, outils collaboratifs, proxy
editing, templates...)

Sophian Veri: effet,transition en general je me sert que de ça

\paragraph{}
6-  Quels fonctionnalités considérez vous comme indispensable, (même si vous
ne les utilisé pas au quotidien)?
Le rognage de final cut sur premiere il est mal conçu et c'est très
énervant.

\paragraph{}
7 Quel pourcentage des fonctionnalité du logiciel pensez vous utilisez
en tout?
Sophian Veri: Je dirais 30-35%

\paragraph{}
8- Seriez vous prêts à utilisez des logiciels ayant moins de fonctionnalités,
mais qui répondrais de manière plus efficace à vos besoins?
Sophian Veri: cela dépend si il manque des trucs dont je ne me
suis jamais servis je m'est égale. Le problème c'est
trouver lesquels car au final mieux en avoir trop que pas assez.

\paragraph{}
9-  Le prix du logiciel est-il un critère de choix selon vous?
Sophian Veri: oui

\paragraph{}
10- Avez-vous des problèmes de stabilité (de bugs)?
Sophian Veri: non pas trop de bug mais ce qui est ennuyeux c'est les
rendu beaucoup trop long

\paragraph{}
11- Connaissez vous certains logiciels libre d'édition vidéo?
Sophian Veri: Non, mais il faut que j'essaye

\subsection {Interview de Yves Faure,  responsable technique à TL7
(Télévision de Saint Étienne)}

\paragraph{}
1-  Quel logiciel d'édition vidéo utilisez vous à l'heure actuel ?
Yves Faure: Pour le montage video, principalement Final cut pro il peu
arriver dans de très rare cas que l'on utilise adobe premiere. Notre
parque informatique est basé sur mac. En ce qui concerne l'habillage et
l'infographie on utilise photoshop et  after effect pour les effets (bien
que dans de nombreux cas, on fasse les effets directement dans Final Cut).

\paragraph{}
2- Quel format de vidéo produisez vous?
Un peu de tout: Reportage, documentaire, plateau magazine, film de reportage, spot
publicitaire, captation musique et théâtral.

\paragraph{}
3- Ce logiciel répond t il à tout vos besoin en terme de montage?
Oui, largement. Nous avons des besoins spécifique en terme d'organisation,
archivage, gestion de sous-titrage, mais cela sort du scope du logiciel de
montage.

\paragraph{}
4- Quels défauts vous viennent à l'esprit quand vous penser à cette outil?
Le gros problème qui me vient à l'esprit est le fait que Final Cut 10 soit
extrêmement osé. Apple q décidé de revoir complètement l'interaction
utilisateur et cela va nous faire perdre du temps (et donc de l'argent). 

Il y a aussi de petits default d'ergonomie qui sont irritants.

Le fait qu'il soit aussi puissant est pour nous un default puisque cela
complexifie la tache du monteur.

Son prix très élevé est aussi un problème pour notre structure (bien que bien
moins chère que d'autres concurrents.

Par default, les fichier de rendu video, le cache de vignette, est stocké dans
le dossier final cut pro global au  système et pas avec le projet, ce qui
signifie que l'on doit changer cela à chaque fois et une fois de plus c'est une
grosse perte de temps.

Lorsqu'il y q des rupture de timecodes dans les fichier, le logiciel réagit mal;
et cela est source de problème régulièrement.

La gestion des formats est assez mauvaise.

Trop configurable, a tout les moment de l'édition.

Pas de sortie moniteur direct chez apple. Pour visualiser le rendu final sur les
moniteur et ainsi être sur de la qualité du montage (en particulier au niveau de
la lumière et des couleur, on est obligé d'effectuer le rendu et ensuite
seulement le voir sur les moniteurs dédiés. On devrai pouvoir brancher nos mac
sur les moniteur et regarder en temps réel le résultat final.

\paragraph{}
5- Quels qualité fait que vous êtes satisfait de ce logiciel?

La dernière version de Final cut permet le réetalonage automatique (de l'image
et du son), cela va vraiment faciliter le travail des monteurs.

Le fait qu'il s'agit du standard actuel dans le milieu est très important pour
nous. Cela nous permet de communiquer facilement avec nos confrère.

Le fait que l'on ai les effets directement intégré dans le logiciel nous
permet d'accélérer le montage dans de nombreux cas.

Dans le cadre de magazines et films publicitaire, on utilise beaucoup les
animation (transformations) tel que la modification de l'échelle de l'image,
le rognage. Aussi, le lissage des bords et les ombres portés de l'image nous
permette régulièrement de faire de montages mieux léchés.


\paragraph{}
6- Pour vous, finalcut est la meilleur solution de montage?
Oui, c'est sure

\paragraph{}
7-  Quels fonctionnalité utilisez vous au quotidien (ex: Multicamera, effets,
transitions, keyframes, time remapping, outils collaboratifs, proxy
editing, templates...)

Au quotidiens, nous utilisons: Transition, effet, mixage audio, montage cut,
retouche de couleurs, étalonnage, colorimétrie, multi-images, Gestion de la
vitesse par clip entier, pas de time remapping a proprement parlé.

\paragraph{}
8-  Quels fonctionnalités considérez vous comme indispensable, (même si vous
ne les utilisé pas au quotidien)?

Je pense que le multicamera (pour la couverture d'évènement tel que les
concerts) est la fonctionnalité importante même si pas utilisé au quotidien.

\paragraph{}
9- Quel pourcentage des fonctionnalité du logiciel pensez vous utilisez
en tout?

En general 10%, jusqu'à 50% maximum.

\paragraph{}
10- Seriez vous prêts à utilisez des logiciels ayant moins de
fonctionnalités, mais qui répondrais de manière plus efficace
à vos besoins?

Non, car très nous sommes très attaché à la connaissance du logiciel.
Changer de logiciel voudrais dire, 40 personnes à former, changer les
habitudes et cela est très complexe, en particulier pour les professionnels
du montage vidéo! Nous avons une solution qui nous conviens et qui est leader
sur le marché, il faudrait une vrai évolution du marché pour que l'on pense à
changer.

\paragraph{}
11-  Le prix du logiciel est-il un critère de choix selon vous?

Oui et non, c'est chère mais ca marche. Autant ce donner les moyens pour avoir
un produit qui nous permet d'en gagner par la suite.

\paragraph{}
12- Avez-vous des problèmes de stabilité (de bugs) avec finalcut?

Non, la stabilité est un point fort de Final Cut.

\paragraph{}
13- La dépendance vis-à-vis du créateur du logiciel que vous utilisez
vous paraît il être quelque chose de dangereux ?

Non, on s'arrange comme on peu. La version que l'on utilise actuellement
nous convient, si il faut que l'on continu avec celle-ci, on le ferra.
