\newpage

\chapter*{Annexes}
\section*{Interview de Sophian Veri,
Responsable Post Production chez Be Movie, France}

\paragraph{}
1-  Quelles logicielles d'éditions vidéos utilisez-vous à l'heure actuelle?
Sophian Veri: J'utilise exclusivement finalcut

\paragraph{}
2- Quels formats de vidéos produisez vous?
Sophian Veri: Je produis des clips, reportages, pubs et courts métrages

\paragraph{}
3- Ce logiciel répond-t-il à tous vos besoins en terme de montage?
Sophian Veri: généralement oui

\paragraph{}
4- Quels défauts vous viennent à l'esprit quand vous pensez à cette outil?
Sophian Veri: le montage multicamera est mal géré, le temps de rendu trop long car le
logiciel ne fonctionne pas avec toute la mémoire vive de l'ordinateur
contrairement à la suite Adobe.

\paragraph{}
5- Considérez-vous Final Cut Pro comme étant la meilleure solution de montage,
si oui, pourquoi?
Sophian Veri: Oui, sans aucun doute. Pour sa facilité d'utilisation,
les codecs acceptés y sont nombreux et la liste des effets est
longue\ldots

\paragraph{}
6-  Quelles fonctionnalités utilisez vous au quotidien (ex: Multicamera, effets,
transitions, keyframes, time remapping, outils collaboratifs, proxy
editing, Templates...)

Sophian Veri: effets de mauvais téléviseur, time remap, cache patate (qui m'évite
de passer par after.) ralenti, fondu enchainé

Thibault Saunier: Qu'est-ce que cache patate?
Sophian Veri: cache patate, c'est masque dans after

\paragraph{}
7-  Quelles fonctionnalités considérez-vous comme indispensables, (même si vous
ne les utilisez pas au quotidien)?
Sophian Veri: la modification des couleurs,  tout ce qui est travail de
l'image, contraste, netteté, saturation.

\paragraph{}
8- Quel pourcentage de fonctionnalités du logiciel pensez-vous utiliser
en tout?
Sophian Veri: il y a  tellement de fonctionnalités que je serais tenté
de dire 25%

\paragraph{}
9- Seriez vous prêts à utiliser des logiciels ayant moins de fonctionnalités,
mais qui répondraient de manière plus efficace à vos besoins?
Sophian Veri: pourquoi pas à condition qu'ils soient aussi intuitifs et que les effets
que j'utilise soient tout aussi bien gérés

\paragraph{}
10-  Le prix du logiciel est-il un critère de choix selon vous?
Sophian Veri: oui, en tant que nouvelle jeune entreprise, le prix est
un critère de choix

\paragraph{}
11- Avez-vous des problèmes de stabilité (de bugs) avec finalcut?
Sophian Veri: les bugs, assez rarement

\paragraph{}
12- La dépendance vis-à-vis du créateur du logiciel que vous utilisez
vous parait-elle être quelque chose de dangereux?

Sophian Veri: oui dans le sens où on ne sait jamais quelles transformations le
logiciel subira avec la version suivante. Il arrive que la version soit moins adaptés
et qu'elle ne me donne plus toute satisfaction. C'est le cas il me semble de Final Cut
X qui a l'air d'être raté.

\paragraph{}
13- Connaissez-vous certains logiciels libres d'édition vidéo?
Sophian Veri: Je ne sais pas vraiment ce que c'est.

Thibault Saunier: Merci bien d'avoir pris le temps de répondre

Sophian Veri: mais de rien ...

\section*{Interview de Karim Hachemi, Monteur chez Falfyprod, Film d'entreprise,
Court Métrages et film de mariage, France}

\paragraph{}
1-  Quel logiciel d'édition vidéo utilisez vous à l'heure actuel?
Karim Hachemi: Final cut mais surtout Adobe CS5

\paragraph{}
2- Quel types de vidéo produisez vous?
Karim Hachemi: Clips, Pubs et institutionnels.

\paragraph{}
3- Ce logiciel répond-t-il à tous vos besoins en terme de montage?

Karim Hachemi: Adobe Première non, mais Final Cut oui.
Nous somme obligé d'utiliser Adobe première du fait que
le format produit par la camera Canon 5D est mal supporté par Final Cut.
Nous aurions donc besoin de les convertir et nous perderions beaucoup de
temps, c'est pourquoi nous utilisons la suite adobe bien qu'elle ne nous donne
pas complètement satisfaction.

\paragraph{}

4- Quels défauts vous viennent à l'esprit quand vous pensez à cet outil?

Karim Hachemi: Final cut est un tout petit peu plus compliqué mais
est plus complet.


\paragraph{}
5-  Quels fonctionnalités utilisez-vous au quotidien (ex: Multicamera, effets,
transitions, keyframes, time remapping, outils collaboratifs, proxy
editing, Templates...)

Karim Hachemi: Effets et transitions, en général je me sers que de ça.

\paragraph{}
6-  Quelles fonctionnalités Considérez-vous comme indispensables, (même si vous
ne les utilisez pas au quotidien)?
Karim Hachemi: le rognage de final cut sur première est mal conçu et c'est très
énervant.

\paragraph{}
7 Quel pourcentage des fonctionnalités du logiciel pensez-vous utiliser
en tout?
Karim Hachemi: Je dirais 30-35%

\paragraph{}
8- Seriez-vous prêts à utilisez des logiciels ayant moins de fonctionnalités,
mais qui répondraient de manière plus efficace à vos besoins?
Karim Hachemi: cela dépend s'il manque des fonctionnalités dont je ne me
suis jamais servi cela m'est égal.

\paragraph{}
9-  Le prix du logiciel est-il un critère de choix selon vous?
Karim Hachemi: oui

\paragraph{}
10- Avez-vous des problèmes de stabilité (de bugs)?
Karim Hachemi: non pas trop de bugs mais ce qui est ennuyeux c'est les
rendus beaucoup trop long

\paragraph{}
11- Connaissez-vous certains logiciels libre d'édition vidéo?
Karim Hachemi: Non, mais il faut que j'essaie

\section*{Interview de Yves Faure, responsable technique chez TL7
(Télévision de Saint Étienne)}

\paragraph{}
1-  Quel logiciel d'édition vidéo utilisez-vous à l'heure actuelle ?
Yves Faure: Pour le montage video, j'utilise principalement Final cut pro. Il
arrive dans de très rares cas que l'on utilise adobe premiere. Notre
parc informatique est basé sur mac. En ce qui concerne l'habillage et
l'infographie on utilise ``photoshop'' et  ``After Effect'' pour les effets (bien
que dans de nombreux cas, on fasse les effets directement dans Final Cut).

\paragraph{}
2- Quel format de vidéo produisez-vous?
Yves Faure: Un peu de tout: Reportages, documentaires, plateaux magazines, films de reportage, spots
publicitaires, captations musique et théâtral.

\paragraph{}
3- Ce logiciel répond-t-il à tout vos besoins en terme de montage?
Yves Faure: Oui, largement. Nous avons des besoins spécifiques en terme d'organisation,
archivage, gestion de sous-titrage, mais cela sort du scope du logiciel de
montage.

\paragraph{}
4- Quels défauts vous viennent à l'esprit quand vous pensez à cet outil?
Yves Faure: Le problème important qui me vient à l'esprit est le fait que Final Cut 10 soit
extrêmement osé. Apple a décidé de revoir complètement l'interaction
utilisateur et cela va nous faire perdre du temps (et donc de l'argent).

Il y a aussi de petits défauts d'ergonomie qui sont irritants.

Le fait qu'il soit aussi puissant est pour nous un défaut puisque cela
complexifie la tâche du monteur.

Son prix très élevé est aussi un problème pour notre structure (bien que bien
moins cher que d'autres concurrents).

Par défaut, les fichiers de rendu video, le cache de vignette, est stocké dans
le dossier final cut pro global au  système et pas avec le projet, ce qui
signifie que l'on doit changer cela à chaque fois et une fois de plus c'est une
perte de temps importante.

Lorsqu'il y q des ruptures de timecodes dans les fichiers, le logiciel réagit mal
et cela est régulièrement une source de problème.

La gestion des formats est assez mauvaise.

Trop configurable en tout points.

Pas de sortie moniteur directe chez Apple. Pour visualiser le rendu final sur les
moniteurs et ainsi être sur de la qualité du montage (en particulier au niveau de
la lumière et des couleurs, on est obligé d'effectuer le rendu et ensuite
seulement le voir sur les moniteurs dédiés. On devrait pouvoir brancher nos Mac
sur les moniteurs et regarder en temps réel le résultat final.

\paragraph{}
5- Quelles sont les qualités apportées par ce logiciel et qui vous donne satisfaction?

Yves Faure: La dernière version de Final cut permet le réetalonage automatique (de l'image
et du son), cela va vraiment faciliter le travail des monteurs.

Le fait qu'il s'agit du standard actuel dans le milieu est très important pour
nous. Cela nous permet de communiquer facilement avec nos confrères.

Le fait aussi que l'on ait les effets directement intégrés dans le logiciel nous
permet d'accélérer le montage dans de nombreux cas.

Dans le cadre de magazines et films publicitaires, on utilise beaucoup les
animations (transformations) tel que la modification de l'échelle de l'image,
le rognage. Aussi, le lissage des bords et les ombres portés de l'image nous
permettent régulièrement de faire des montages mieux léchés.


\paragraph{}
6- Pour vous, finalcut est la meilleure solution de montage?
Yves Faure: Oui, c'est sûr.

\paragraph{}
7-  Quelles fonctionnalités utilisez-vous au quotidien (ex: Multicamera, effets,
transitions, keyframes, time remapping, outils collaboratifs, proxy
editing, Templates...)

Yves Faure: Au quotidien, nous utilisons: Transition, effet, mixage audio, montage cut,
retouche de couleurs, étalonnage, colorimétrie, multi-images, gestion de la
vitesse par clip entier, pas de time remapping à proprement parlé.

\paragraph{}
8-  Quelles fonctionnalités considérez-vous comme indispensables, (même si vous
ne les utilisez pas au quotidien)?

Yves Faure: Je pense que le multicamera (pour la couverture d'évènement tel que les
concerts) est la fonctionnalité importante, même si elle n'est pas utilisée au quotidien.

\paragraph{}
9- Quel pourcentage des fonctionnalités du logiciel pensez-vous utiliser
en tout?

Yves Faure: En général 10\%, jusqu'à 50\% maximum.

\paragraph{}
10- Seriez-vous prêts à utiliser des logiciels ayant moins de
fonctionnalités, mais qui répondrait de manière plus efficace
à vos besoins?

Yves Faure: Non, car nous sommes très attachés à la connaissance du logiciel.
Changer de logiciel signifie 40 personnes à former, changer les
habitudes et cela est très complexe en particulier pour les professionnels
du montage vidéo! Nous avons une solution qui nous convient et qui est leader
sur le marché, il faudrait une vrai évolution du marché pour que l'on pense à
changer.

\paragraph{}
11-  Le prix du logiciel est-il un critère de choix selon vous?

Yves Faure: Oui et non, c'est cher mais ça marche. Autant se donner les moyens pour avoir
un produit qui nous permet de gagner par la suite.

\paragraph{}
12- Avez-vous des problèmes de stabilité (de bugs) avec Final Cut?

Yves Faure: Non, la stabilité est un point fort de Final Cut.

\paragraph{}
13- La dépendance vis-à-vis du créateur du logiciel que vous utilisez
vous paraît-elle être quelque chose de dangereux ?

Yves Faure: Non, on s'arrange comme on peut. La version que l'on utilise actuellement
nous convient. S'il faut continuer avec celle-ci, on le fera.
